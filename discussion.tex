\section{Discussion}
\label{sec:discussion}

\subsection{More Dimensions} 

In this paper, we discuss prior works along just one dimension, the trust
between the application and OS. In this dimension, we divide prior works into
"App does not trust OS", "OS does not trust App" and "Mutually Mistrusting".
There are also other dimensions to discuss the prior works. For instance, some
works \cite{Flicker, TrustVisor, Overshadow, InkTag, VirtualGhost,
CloudTerminal, SFI, PittSFIeld, NaCl, Krude, MiniBox} are proposed for PC and
some works \cite{TLR, VeriUI, TrustUI, TrustDroid, Cells, AirBag} are proposed
for smartphone. While some works are proposed for a local PC or smartphone, some
works are proposed for cloud environment, such as PaaS \cite{Krude, MiniBox}.
Flicker and AirBag only work for single application, others such as TrustVisor,
Cells and MiniBox can support many applications from different users
simultaneously. Some works are designed to support the protection and isolation
of just pieces of application logic (PAL), while Overshadow and AirBag's
protection mechanisms are in the granularity of the whole application.
Overshadow does not require the modification to the source code of the
application while InkTag requires the user to change the way of programming.

\subsection{Limitations}

Although many prior works have already solved the problems with a variety of
mechanisms, there still exist some limitations. While MiniBox \cite{MiniBox} is
the first known attempt to remove the trust between the application and OS on
PaaS, it supports only a single guest OS at this time. Besides, There is no two-way
protection on Android. Android makes the problem more complex. Unlike Linux,
where one task is implemented as a single application, the task on Android is
usually accomplished by a set of applications together. The compromised Android
runtime is able to infect the application in a way that is similar to Iago
attack. It is expected that the Android application cannot be compromised even
its underpinning runtime is malicious. We hope to investigate how Linux kernel
helps verify the behavior of a compromised Android runtime. 

The hypervisor-based isolation solutions, including TrustVisor, InkTag and
MiniBox, cause overhead in context switch. The VMFUNC instruction released on
the latest Intel 4th Generation Processor enables the software in guest OS to
switch the hardware Extended Page Table (EPT) without the VM exit. Since VM
exit is one of primary reasons for performance overhead of VM, we hope the investigation on
how to perform secure environment switch using the VMFUNC instruction will
improve the performance of hypervisor based solution.
