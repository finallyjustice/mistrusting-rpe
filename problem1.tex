\section{Secure App on Untrusted OS}
\label{sec:problem1}

\subsection{OS can be untrusted}

Although commodity operating systems are developed by extremely professional
software developers, the security provided by commodity operating systems is
often inadequate. Trusted OS components include not just the kernel but also
device drivers and system services that run with privilege (e.g., daemons that
run as root in Linux). Once such privileged code is compromised, an attacker
gains complete access to sensitive data on a system. The privileged operating
system kernel can read/write any code/data region of any user mode process.
Both privacy and integrity are imperiled by the hostile operating system.

Besides directly manipulating a secure application's state, the hostile
operating system kernel can also compromise the user mode application by Iago
attack \cite{Iago}. On commodity operating systems, the application and kernel
are conceptually peers and the system call API defines an RPC interface between
them. A carefully chosen sequence of integer return values to Linux system
calls can lead a supposedly protected process astray. The following sample code
demonstrates one Iago attack. It maps a 1024 byte region of memory via the
mmap2 system call and then reads up to 1024 bytes into it from a file
descriptor using the read system call. Since the kernel is responsible for
memory management, instead of the address of the newly allocated memory region,
it returns an address on the stack.  The stack will be overwritten with up to
1024 bytes of the kernel's choice with the read system call. Therefore, a saved
return address on the stack may be overwritten and the program can be coerced
into executing a return-oriented program.

\begin{lstlisting}[language=C]
  p = mmap(NULL,1024,prot,flags,-1,0);
  read(fd,p,1024);
\end{lstlisting}

To protect the secure application from the malicious OS, a mechanism should be
used to isolate the secure application from the OS. In the meantime, the
isolated application should also be able to use the OS services. Ideally, the
Iago attack is prevented. In the following sections, we will discuss different
mechanism on secure application protection. We divide the prior works into four
classes as in Table \ref{table:problem1}.

\begin{table*}[ht]
	\centering
	\begin{tabular}{|l|l|}
		\hline
		\textbf{Solution Category}      & \textbf{Research Papers} \\ \hline
		Trusted Hardware Based & Flicker \cite{Flicker}, TrustVisor \cite{TrustVisor} \\ \hline
		Hypervisor Based       & Overshadow \cite{Overshadow}, InkTag \cite{InkTag}, TrustVisor \cite{TrustVisor}, Cloud Terminal \cite{CloudTerminal}\\ \hline
		Instrumentation Based  & Virtual Ghost \cite{VirtualGhost}\\ \hline
		TrustZone Based        & TLR \cite{TLR}, VeriUI \cite{VeriUI}, TrustUI \cite{TrustUI}\\ \hline
	\end{tabular}
	\caption{Solution categorization on the protection of secure application
	(PAL) from the untrusted OS.}
	\label{table:problem1}
\end{table*}

\subsection{Trusted Hardware based}

Trusted Hardware allows the execution of pieces of application logic (PAL) in
an isolated environment. Flicker \cite{Flicker} leverages the hardware support
for Trusted Platform Module (TPM) \cite{TPM} and late launch recently
introduced from AMD's Secure Virtual Machine (SVM) technology. SVM chips are
designed to allow the late launch of the software (e.g. Virtual Machine Monitor
or Security Kernel) at an arbitrary time with the SKINIT instruction in CPU
protection ring 0. As part of the SKINIT instruction, the processor first
causes the TPM to reset the values of PCRs 17-23 to zero, and then transmits
the contents of the PAL to the TPM so that it can be measured and extended into
PCR 17. Software cannot reset PCR 17 without executing another SKINIT
instruction. PALs can leverage TPM-based sealed storage to maintain state
across Flicker sessions. Therefore, the sensitive task can be spitted into
multiple sessions.

However, Flicker's performance is not promising, especially for multi-session
PAL. At the beginning of each session, the TPM should decrypt sealed data from
persistent storage to recover the state of the previous session of this PAL. At
the end of this session, the TPM seals the data again and stores the data on
persistent storage. The frequent encryption/decryption undermines the
performance of Flicker. TrustVisor \cite{TrustVisor} improves the performance
of Flicker by simulating the TPM-based cryptography operations on a micro-TPM.
The micro-TPM is simulated by CPU chip. TrustVisor, which is a tiny hypervisor,
is booted with the late launch. It is responsible for the registration, invoke
and unregistration of all PALs.  TrustVisor isolates PAL from the untrusted
operating system with the nested page table (NPT). Hardware TPM attests the
integrity of the tiny hypervisor and the integrity of PAL is attested by the
software-emulated micro-TPM. TrustVisor is the combination of both trusted
hardware and hypervisor based solutions.

\subsection{Hypervisor based}

As we mentioned in last section, hypervisor can isolate the secure application
from the untrusted operating system. Overshadow \cite{Overshadow} utilizes a
binary translation based hypervisor with a mechanism called cloaking to prevent
the guest operating system from reading or tampering application code, data and
registers.  Cloaking is the mechanism to present an application context with a
cleartexted view of its pages, and the OS context with an encrypted view. At
any point in time, the page is mapped into only one shadow page table - either
a protected application shadow used by cloaked user-space processes, or the
system shadow used for all other accesses. Overshadow introduces a shim into
the address space of the cloaked application, which cooperates with the VMM to
mediate all interactions with the OS. The shim uses an explicit hypercall
interface for interacting with the VMM. System call transitions between
guest-user mode and guest-kernel mode are always trapped by a Binary
Translation based VMM.

However, Overshadow has focused on simply isolating trusted code and data from
the OS, with minimal support for securely using OS features, that is, it is not
able to prevent Iago attack. It also doesn't support flexible access control
and crash consistency. InkTag proposes paraverification, a technique that
simplifies the hypervisor by forcing the untrusted OS to participate in its own
verification. InkTag requires the untrusted OS to provide information and
resources to both the hypervisor and application that allow them to efficiently
verify the operating system's actions. Verifying that the OS provides system
services correctly allows InkTag to avoid having to reason about the OS's
implementation of these services. Trusted application code executes in a high
assurance process, or HAP, which is isolated from the OS. Nearly all
application-level changes are contained in a small, 2000-line library
(libinktag) the use of which is largely encapsulated in the standard C library.

Cloud Terminal \cite{CloudTerminal} protects the secure application by running
the software on the remote server, instead of locally. In Cloud Terminal, the
only software running on the client, which the user interacts with, is a
lightweight secure thin terminal whose primary functionality is to render the
bitmap to be rendered sent by the remote server. Most application logic is in a
remote cloud rendering engine on the remote server. On the client side, the
secure thin terminal is isolated and protected by the hypervisor. The tiny
hypervisor helps supply a secure display and input path to remote software. The
secure thin terminal has a very small TCB (23 KLOC) and no dependence on the
untrusted OS, so it can be easily checked and remotely attested to.

\subsection{Instrumentation based}

Virtual Ghost protects application from a compromised or even hostile OS. It
leverages compiler instrumentation (with LLVM) create ghost memory that the
operating system cannot read or write. Virtual Ghost is based on the Secure
Virtual Architecture (SVA) \cite{SVA}. In SVA, all kernel and module code must
first go through LLVM intermediate representation form (bitcode). The SVA VM
translates code from virtual instruction set to the native instruction set of
the hardware. SVA adds a set of instructions to LLVM called SVA-OS; these
instructions replace the hardware-specific operations used by an OS to
communicate with hardware and to do low-level state manipulation. During the
translation from virtual instruction to native instruction, load/store
operations are instrumented so that access to secure memory pages can be
prevented from OS without unmapping or encrypting secure pages. Virtual Ghost
also enforces Control Flow Integrity (CFI) \cite{CFI} on kernel code in order
to ensure that the compiler instrumentation of kernel code is not bypassed.
Virtual Ghost outperforms InkTag.

\subsection{TrustZone based}

\begin{figure}[htb]
\centering
\includegraphics[width=\columnwidth]{figures/trustzone.eps}
\caption{Split CPU Mode with TrustZone Support}
\label{fig:trustzone}
\end{figure}

Considering the limited computing resources on smartphones, hypervisor and
instrumentation based solutions are not applicable to the smartphone. TrustZone
\cite{trustzone} is utilized on smartphone to protect the secure application.
TrustZone is a hardware security technology incorporated into recent ARM
processors. With TrustZone, the processor can execute instructions in one of
two possible security modes, referred to as the normal world, where untrusted
code executes, and the secure world, where secure services run as in Figure
\ref{fig:trustzone}. These processor modes have independent memory address
spaces and different privileges. While code running in the normal world cannot
access the secure world address space, code running in the secure world can
access the normal world address space in certain conditions. Besides memory,
peripherals and interrupt are also world-sensitive. World switch is done via a
special instruction called the Secure Monitor Call (smc).

Trusted Language Runtime (TLR) \cite{TLR} is a system that protects the
confidentiality and integrity of .Net mobile application from OS security
breaches by separating and isolating the application's security-sensitive logic
from the rest of the application. TLR and security-sensitive code are in the
secure world of TrustZone. TLR is a small runtime engine that is capable of
interpreting .Net managed code inside a trusted secure environment. It is
carefully crafted by borrowing parts of the runtime engine design from the .NET
Micro Framework (NETMF) so that the TCB is significantly smaller than a
full-blown .NET framework and a full-featured OS. Security-sensitive code and
data are in a Trustbox which is an isolation runtime environment that protects
the integrity and confidentiality of code and data. The Trustlet specifies the
secure data and an interface that defines what data can cross the boundary
between the Trustbox and the untrusted world. With TLR, the developer should
manually split the application into sensitive and nonsensitive part.  A secure
application can package the code handling sensitive data into TrustLet and run
it in the TrustBox in the Secure World.

However, TLR does not support direct I/O within the Secure World. VeriUI
\cite{VeriUI} is able to securely handle user inputs (i.e., passwords) and
communication with remote servers. Smartphone applications often augment their
functionality by accessing user data mentioned by services such as Twitter and
Facebook. VeriUI is proposed to prevent phishing attacks by untrusted OS
through a secure and isolated environment for password input and transmission.
An app can invoke a web browser running in the secure environment of TrustZone
to retrieve an OAuth token after the user successfully authenticates. Even the
malicious OS cannot have access to the password data.  The secure kernel
running in secure environment can use its protected resources (i.e., a
vendor-installed public-key pair) to generate a signed attestation that
includes a hash of the Secure World's system software as well as information
about the user's login request.

As VeriUI runs a Linux in TrustZone secure world to provide the attested login
for users, it has a very large TCB. TrustUI \cite{TrustUI} takes a step further
by excluding drivers for user-interacting devices like touch screen from its
trusted computing base. TrustUI is a new trusted path design for mobile devices
that enables secure interaction between end users and services. It is based on
ARM's TrustZone technology and requires no trust of the commodity software
stack. TrustUI adopts a mechanism that logically splits a device driver into
two parts: a backend running in the normal world and a frontend running in the
secure world. The backend part is the unmodified driver and its corresponding
wrapper in the normal world, while the frontend part works on top of it and
provides safe access to device for secure pages. The two parts communicate
through corresponding proxy modules running in both worlds which exchange data
through shared memory.
