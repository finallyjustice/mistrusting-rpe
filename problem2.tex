\section{Untrusted App on Benign OS}
\label{sec:problem2}

In this section, we discuss the protection of the operating system from an
untrusted application or a piece of untrusted code. In this paper, we call both
untrusted application and untrusted code as untrusted module. The untrusted
module can be pieces of native code downloaded by a web browser, an application
uploaded and executed on the PaaS server, or an Android application downloaded
from an untrusted third-party website. Although the isolation of untrusted module can
prevent it from infecting the operating system, it is far from enough. There are
other challenges. First, the isolated code module wants to interact with the
operating system services via system calls. Second, the isolation (sandbox)
should not impact the performance of program execution. Third, a low
implementation overhead is expected, that is, the modification to compiler,
linker, application source code and operating system kernel source code should
be minimized. Last, since the smartphone has limited resources, the isolation
should be lightweight. 

In this paper, we categorize the prior works according to the granularity of
isolation as in Table \ref{table:problem2}. The granularity of isolation varies,
including intra-process, inter-process, inter-namespace and inter-VM. 

\begin{table*}[ht]
	\centering
	\begin{tabular}{|l|l|}
		\hline
		\textbf{Solution Category}      & \textbf{Research Papers} \\ \hline
		Inter-VM Based         & KVM/ARM \cite{KVM/ARM} \\ \hline
		Intra-Process Based    & SFI \cite{SFI}, PittSFIeld \cite{PittSFIeld}, Native Client \cite{NaCl} \\ \hline
		Inter-Process Based    & Native Client \cite{NaCl}, Krude \etal \cite{Krude} \\ \hline
		Inter-Namespace Based  & TrustDroid \cite{TrustDroid}, Cells \cite{Cells}, AirBag \cite{AirBag} \\ \hline	
	\end{tabular}
	\caption{Solution categorization on the protection of OS from the untrusted
	application.}
	\label{table:problem2}
\end{table*}

\subsection{Inter-VM based}
\label{sec:problem2:inter-vm}

The naive approach is to isolate each untrusted module into its corresponding
VM. There are a variety of virtual machine monitors, including Xen \cite{xen},
KVM \cite{kvm}, Qemu \cite{qemu}, and VMWare \cite{vmware}.  Recently, the
hardware virtualization extension has been added into the ARM and the ARM based
KVM \cite{KVM/ARM} is integrated into the Linux kernel since Linux 3.9. Since
this approach is clear and self-explained, we will not discuss it in detail in
this paper.

\subsection{Intra-Process based}
\label{sec:problem2:intra-process}

Intra-Process protection is to isolate the untrusted module from other
memory regions in the same address space. SFI \cite{SFI} is proposed to sandbox
the untrusted module by rewriting the untrusted code at the instruction level,
that is, to instrument store/load and control flow instructions. However, it
only works for RISC architectures. PittSFIeld \cite{PittSFIeld} presents
sandboxing technique that can be applied to CISC architecture e.g. IA-32, and
whose application can be checked at load-time to minimize the TCB.  Unlike RISC
architectures, whose instructions have the same length, the x86 has
variable-length instructions that might start at any byte. To avoid this
problem, PittSFIeld divides memory into segments whose size and location is
16-byte aligned.  New instructions are instrumented before store/load and
control flow instructions to check that the sandboxed module cannot read/write
data outside sandbox and transfer to illegal control flow target outside
sandbox.

A weakness of PittSFIeld is it cannot effectively mediate the access from
untrusted module to operating system services. Besides isolating the untrusted
module, Native Client \cite{NaCl} also allows the module to interact with
services, such as file I/O and local database access, by the combination of
intra-process and inter-process approaches.  An Intra-Process based sandbox is
used to isolate the untrusted module from the runtime service, which resides in
the same address space as the sandboxed untrusted module. Runtime service
mediates the communication between the untrusted module and other processes
including web browser and other services. 

\subsection{Inter-Process based}
\label{sec:problem2:inter-process}

Krude \etal \cite{Krude} propose an inter-process based approach to sandbox the
untrusted module. It is especially designed for PaaS architectures, where
code execution needs to be isolated to protect tenants from unauthorized access
to their data by other tenants and to protect the host system from any type of
intrusion by other tenants. The untrusted module is uploaded to the PaaS server
and it is isolated in a new process. Krude \etal use the process barrier and
the seccomp filter mechanism to restrict access to memory and to the system
call interface. Almost all system calls are blocked for the isolated process.
Besides memory allocation and deallocation, the isolated process can communicate with OS
by sending to request to a supervisor process via pipe, which is the IPC
mechanism on Linux. The supervisor process will process the request and send
the response back to the isolated process also via pipe.

\subsection{Inter-Namespace based}
\label{sec:problem2:inter-namespace}

The Inter-Namespace based approach is primary proposed for smartphone running
Android. Nowadays, smartphones are ubiquitous. Many people use the smartphone both for
working and personal needs. However, the personal applications downloaded from
the untrusted website can compromise the application issued by the trusted
enterprise. Therefore, many users carry multiple phones to accommodate work,
personal and geographic mobility need. Cells \cite{Cells} proposes a smartphone
virtualization solution so that multiple virtual smartphones can run
simultaneously on the same physical smartphone in an isolated, secure manner.

Unlike the virtualization techniques mentioned in section
\ref{sec:problem2:inter-vm}, Cells leverages a lightweight OS-level
virtualization by the utilization of namespace. Linux namespace is being used
by OpenVZ \cite{openvz} and LXC \cite{lxc}. A set of processes can be grouped
into the same namespace. Each Linux namespace has PID namespace isolation,
network namespace isolation, UTS namespace isolation, mount namespace isolation
and IPC namespace isolation.

Cells observes that smartphones display only a single application at a time,
and introduces a usage model which has one foreground Virtual Phone (VP) that
is displayed and multiple background VPs that are not displayed at any given
time. The foreground VP is always given direct access to hardware devices while
the background VPs are given shared access to hardware devices when the
foreground VP does not require exclusive access. Cells provides novel
kernel-level and user-level device namespace mechanisms to efficiently
multiplex hardware devices across multiple VPs. Therefore, untrusted
application inside personal VP (namespace) will not be able to compromise the
trusted application inside enterprise VP.

While Cells aims to embrace the emerging Bring-Your-Own-Device (BYOD) paradigm,
each VP is treated equally and the isolation is achieved at the coarse-grained
VP boundary. Unlike Cells, AirBag \cite{AirBag}'s objective is to boost the
smartphone's defense capability against the malware infection by isolating the
untrusted application in the AirBag environment. By dynamically creating an
isolated runtime environment with its own dedicated namespace and virtualized
system resources, AirBag is able to protect the OS from the malicious untrusted
applications, e.g., an Android game repacked with the malware. AirBag creates
and decouples an Application Isolation Runtime (AIR) from the native Android
runtime, which contains Java \& Native Libraries, Application framework (e.g.,
SurfaceFlinger service) and Dalvik virtual machine.  AIR does not need to be
trusted as it might be potentially compromised by untrusted application. AirBag
multiplexes hardware resources between the AIR and native runtime by either
creating a second resource (e.g., memory buffer) or creating a proxy between
runtimes and hardware to mediate access from different runtimes. 


