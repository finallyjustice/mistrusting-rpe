\section{Untrusted App on Benign OS}
\label{sec:problem2}

In this section, we discuss the protection of the operating system from an
untrusted application or a piece of untrusted code. In this paper, we call both
of untrusted application and untrusted code as untrusted module. The untrusted
module can be pieces of native code downloaded by a web browser, an application
uploaded and executed on the PaaS server, or an Android application downloaded
from an untrusted third part. Although the isolation of untrusted module can
prevent it from infecting the operating system, it is far from enough. There are
other challenges. First, the isolated code module wants to interact with the
operating system services via system calls. Second, the isolation (sandbox)
should not impact the performance of program execution. Third, a low
implementation overhead is expected, that is, the modification to compiler,
linker, application source code and operating system kernel source code should
be minimized. Last, since the smartphone has limited resources, the isolation
should be lightweight. 

In this paper, we categorize the prior works according to the granularity of
isolation. The granularity of isolation varies, including intro-process,
inter-process, inter-namespace and inter-VM. 

\begin{table*}[ht]
	\centering
	\begin{tabular}{|l|l|}
		\hline
		\textbf{Solution Category}      & \textbf{Research Papers} \\ \hline
		Inter-VM Based         & KVM/ARM \cite{KVM/ARM} \\ \hline
		Intra-Process Based    & SFI \cite{SFI}, PittSFIeld \cite{PittSFIeld}, Native Client \cite{NaCl} \\ \hline
		Inter-Process Based    & Native Client \cite{NaCl}, Krude \etal \cite{Krude}, TrustDroid \cite{TrustDroid}\\ \hline
		Inter-Namespace Based  & Cells \cite{Cells}, AirBag \cite{AirBag} \\ \hline	
	\end{tabular}
	\caption{Solution categorization on the protection of OS from the untrusted
	application.}
	\label{my-label}
\end{table*}

\subsection{Inter-VM based}

The naive approach is to isolate each untrusted module into its correpsonding
VM. There are a variety of virtual machine monitors, including Xen, KVM, Qemu,
and VMWare.  Recently, the hardware virtualization extension has been added into
the ARM and the ARM based KVM \cite{KVM/ARM} is integrated into the Linux kernel
since Linux 3.9. Since this approach is clear and self-explained, we will not
discuss it in detail in this paper.

\subsection{Intra-Process based}

Intra-Process protection is to isolate the untrusted module from the other
memory regions in the same address space. SFI \cite{SFI} is proposed to sandbox
the untrusted module by rewriting the untrusted code at the instruction level,
that is, to instrument store/load and control flow instructions. However, it
only works for RISC architectures. PittSFIeld \cite{PittSFIeld} presents
sandboxing technique that can be applied to CISC architecture e.g. IA-32, and
whose application can be checked at load-time to minimize the TCB.  Unlike RISC
architectures, whose instructions have the same length, the x86 has
variable-length instructions that might start at any byte. To avoid this
problem, PittSFIeld divides memory into segments whose size and location is
16-byte aligned.  New instructions are instrumented before store/load and
control flow instructions to check that the sandboxed module cannot read/write
data outside sandbox and transfer to illegal control flow target outside
sandbox.

A weakness of PittSFIeld is it cannot effectively mediate the access from
untrusted module to operating system services. Besides isolating the untrusted
module, Native Client \cite{NaCl} also allows the module to interact
with services, such as file I/O and local database access, by the combination of
intra-process and inter-process approaches.  An Intra-Process based sandbox is
used to isolate the untrusted module from the runtime service, which resides in
the same address space as the sandboxed untrusted module. Runtime service
mediates the communication between the untrusted module and other processes
including web browser and other services. This is similar to the Inter-Process
based solution in the next section.

\subsection{Inter-Process based}

Krude \etal \cite{Krude} is an inter-process based isolation approach to sandbox
the untrusted module. It is specifically designed for PaaS architectures, where
code execution needs to be isolated to protect tenants from unauthorized access
to their data by other tenants and to protect the host system from any type of
intrusion by other tenants. The untrusted module is uploaded to the PaaS server
and it is isolated in a new process. Krude \etal uses the process barrier and
the seccomp filter mechanism to restrict access to memory and to the system call
interface. Almost all system calls are blocked for the isolated process. Besides
memory allocation and deallocation, the isolate can communicate with OS by
sending to request to a supervisor process via pipe, which is the IPC mechanism
on Linux. The supervisor process will process the request and send the response
back to the isolated process also via pipe.

\subsection{Inter-Namespace based}

Namespace can be used to group a set of processes together in an isolated
environment. 

