\section{Two-Way Protection}
\label{sec:problem3}

We discussed the protection of secure application from untrusted OS (e.g.,
TrustVisor \cite{TrustVisor}) in section \ref{sec:problem1} and the protection
of benign OS from untrusted application (e.g., Native Client \cite{NaCl}) in
section \ref{sec:problem2}. In this section, we discuss the removing of trust
between the application and operating system. Within my knowledge, currently,
MiniBox \cite{MiniBox} is the first and the only attempt toward a practical
two-way sandbox for x86 native applications by combining TrustVisor and
Native Client.

Platform-as-a-Service (PaaS) is one of the most widely commercialized forms of
cloud computing. According to Google, in 2012, there were 1M active
applications running on Google App Engine \cite{engine}, where untrusted
applications are sent by customers. Therefore, it is critical to protect the
cloud platform from the untrusted applications. Besides cloud provider such as
Google, security on PaaS is also a concern for cloud customers. People should
rethink the security model of PaaS cloud computing because a two-way sandbox is
desired.

Although it seems promising to combine a one-way sandbox (e.g., TrustVisor) and
a two-way memory isolation mechanism (e.g., Native Client) to establish two-way
protection, there are many challenges. First, a deliberate system design is
required. Second, the interface between software modules for OS protection and
the application should be minimized and secure. Finally, the design of
TrustVisor doesn't support Iago attack prevention. The final system design
should be able to protect applications against Iago attack.

\begin{figure}[htb]
\centering
\includegraphics[width=\columnwidth]{figures/minibox.eps}
\caption{MiniBox Architecture}
\label{fig:minibox}
\end{figure}

MiniBox \cite{MiniBox} combines the one-way sandbox for x86 native code and
hypervisor-based two-way memory isolation. As in figure \ref{fig:minibox}, the
sandbox is split into service runtime modules and OS protection modules. The
service runtime is included in the isolated memory space with the application
together to support application execution. The original disassembly based
sandbox is not required anymore because the hypervisor not only isolates the
application from OS, but also isolates the OS from the application. To prevent
Iago attacks, the system calls are divided into sensitive calls and
non-sensitive calls. All sensitive calls are handled directly by a LibOS
\cite{LibOS}, which the application trusts, residing at the Mutually Isolated
Execution Environment (MIEE).  Non-sensitive calls will be forwarded to the
Regular Environment (RE). The OS in RE handles the non-sensitive calls mediated
by the OS protection module in RE. 
